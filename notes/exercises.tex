\documentclass[a4paper,11pt]{article}

\usepackage[T1]{fontenc}
\usepackage[utf8]{inputenc}
\usepackage{amsmath,amssymb,amsthm,textcomp}

\usepackage[
pdftitle={Topological Exercises}, 
pdfauthor={Lucas Moschen, Fundação Getulio Vargas},
colorlinks=true,linkcolor=blue,urlcolor=blue,citecolor=blue,bookmarks=true,
bookmarksopenlevel=2]{hyperref}
\usepackage{enumerate}
\usepackage{multicol}
\usepackage{setspace}
\onehalfspacing

\usepackage{graphicx}
\usepackage{xcolor}
\usepackage{tikz}
\usepackage{geometry}
\geometry{left=25mm,right=25mm,%
bindingoffset=0mm, top=20mm,bottom=20mm}

\newcommand{\linia}{\rule{\linewidth}{0.5pt}}

% custom theorems if needed
\newtheoremstyle{mytheor}
    {1ex}{1ex}{\itshape}{0pt}{\scshape}{.}{1ex}
    {{\thmname{#1 }}{\thmnumber{#2}}{\thmnote{ (#3)}}}

\theoremstyle{mytheor}
\newtheorem{definition}{Definition}[subsection]

\theoremstyle{mytheor}
\newtheorem{exercise}{Exercise}

\theoremstyle{remark}
\newtheorem*{remark}{Remark}

\newcommand{\T}{\mathcal{T}}
\newcommand{\U}{\mathcal{U}}
\newcommand{\B}{\mathcal{B}}
\newcommand{\Z}{\mathbb{Z}}
\newcommand{\R}{\mathbb{R}}


% my own titles
\makeatletter
\renewcommand{\maketitle}{
    \begin{center}
        \vspace{2ex}
        {\huge \textsc{\@title}}
        \vspace{1ex}
        \\
        \linia\\
        \@author \hfill \@date
        \vspace{4ex}
    \end{center}
}
\makeatother

%%%----------%%%----------%%%----------%%%----------%%%

\title{Topological Data Analysis - Exercises}

\author{Lucas Moschen, EMAp/FGV}

\date{\today}

\begin{document}

\maketitle

\section{General topology}

\subsection{Important definitions}

\begin{definition}
    A topological space is a pair $(X, \T)$ where $X$ is a set and $\T$ is a
    collection of subsets of $X$ such that:
    \begin{enumerate}
        \item $\emptyset \in \T, X \in \T$. 
        \item for every infinite collection $\{O_{\alpha}\}_{\alpha \in A}\subset \T$, we have $\bigcup_{\alpha \in A} O_{\alpha} \in \T$.
        \item for every finite collection $\{O_{i}\}_{1 \le i \le n}
        \subset \T$, we have $\bigcap_{1 \le i \le n} O_{i} \in \T$.
    \end{enumerate}   
\end{definition}

\begin{definition}
    Let $x \in \R^n$ and $r > 0$. The open ball of center $x$ and radius $r$,
    denoted $\B(x, r)$, is defined as: $\B(x, r) = \{y \in \R^n, ||x - y|| < r\}$.
\end{definition}

\begin{definition}
    Let $A \subset \R$ and $x \in A$. We say that $A$ is open
    around $x$ if there exists $r > 0$ such that $\B(x, r) \subset A$. We say
    that $A$ is open if for every $x \in A$, $A$ is open around $x$.
\end{definition}

\begin{definition}
    Let $(X, \T)$ be a topological space, and $Y \subset X$. We define the subspace topology on $Y$ as the following set:
    $$T_{|Y} = \{O \cap Y, O \in \T\}$$
\end{definition}

\begin{definition}
    Let $f : X \to Y$ be a map. We say that $f$ is continuous if for every
    $O \in U$, the preimage $f^{-1}(O) = \{x \in X, f(x) \in O\}$ is in $\T$.
\end{definition}

\subsection{Exercises}

\begin{exercise}
    Let $X = \{0, 1, 2\}$ be a set with three elements. What are the  different topologies that X admits?
\end{exercise}

As we know every Topology contains $\emptyset$ and $\{0,1,2\}$, so we can
disconsider when writing the topologies, that is, all below contain these
subsets. 

\begin{itemize}
    \item (2) Basic: $\{\emptyset, \{0,1,2\}\}$ - $\mathcal{P}(\{0,1,2\})$.  
    \item (8) With $\{0\}$: $\{\{0\}\} - \{\{0\}, \{0,1\}\} - \{\{0\}, \{1,2\}\} -
    \{\{0\}, \{0,2\}\} - \{\{0\}, \{0,2\}, \{0,1\}\}$ \\ 
    $\{\{0\}, \{2\}, \{0,2\}\}- \{\{0\}, \{2\}, \{0,2\}, \{1,2\}\} -
    \{\{0\}, \{2\}, \{0,2\}, \{0,1\}\}$

    \item (8) With $\{1\}$: $\{\{1\}\} - \{\{1\}, \{0,1\}\} - \{\{1\}, \{1,2\}\} -
    \{\{1\}, \{0,2\}\} - \{\{1\}, \{1,2\}, \{0,1\}\}$ \\
    $\{\{0\}, \{1\}, \{0,1\}\} - \{\{0\}, \{1\}, \{0,1\}, \{1,2\}\} -  \{\{0\}, \{1\}, \{0,1\}, \{0,2\}\}$

    \item (8) With $\{2\}$: $\{\{2\}\} - \{\{2\}, \{0,1\}\} - \{\{2\}, \{1,2\}\} -
    \{\{2\}, \{0,2\}\} - \{\{2\}, \{0,2\}, \{1,2\}\}$ \\
    $\{\{1\}, \{2\}, \{1,2\}\} - \{\{1\}, \{2\}, \{1,2\}, \{0,1\}\} - \{\{1\},
    \{2\}, \{1,2\}, \{0,2\}\}$ 
    
    \item (3) No singleton: $\{\{0,1\}\} - \{\{1,2\}\} - \{\{0,2\}\} $
\end{itemize}

\noindent\linia

\begin{exercise}
    Let $\Z$ be the set of integers. Consider the \textit{cofinite
    topology} $\T$ on $\Z$, defined as follows: a subset $O \subset
    \Z$ is an open set if and only if $O = \emptyset$ or $^c O$ is finite. Here, $^cO = \{x \in \Z, x \not\in O\}$ represents the complementary of $O$ in $\Z$
\end{exercise}

\begin{enumerate}
    \item Show that $\T$ is a topology on $\Z$.

    Let's verify the three axioms: 

    \begin{enumerate}
        \item $\emptyset$ is an open set by definition and $\Z$ is open set
        because $^c\Z = \emptyset$ is finite. 

        \item Let $\{O_{\alpha}\}_{\alpha \in A}\subset \T$. So 
        $^cO = ~^c\left(\bigcup_{\alpha \in A} O_{\alpha}\right) = \bigcap_{\alpha
        \in A} ~^cO_{\alpha} \implies ^c O \subset ~^cO_{\alpha}, \forall
        \alpha \in A$. If $\forall \alpha, O_{\alpha} = \emptyset$, then $^cO =
        ~^c\emptyset \implies O = \emptyset$   and $O$ is open. On the other
        hand, if there exists $\alpha \in A$ such that $O_{\alpha} \neq
        \emptyset$ we have $^c O_{\alpha}$ being finite, so is $^c O$, given
        the inclusion. We conclude $O$ is open set. 
        
        \item Let $\{O_{i}\}_{1 \le i \le n} \subset \T$. So 
        $^cO = ~^c\left(\bigcap_{1 \le i \le n} O_i\right) = \bigcup_{1 \le i
        \le n} ~^cO_i$. If $O_i = \emptyset$ for some $1 \le i \le n$, $O =
        \emptyset$ because of the intersection. Alternatively, if $\forall i,
        O_i \neq \emptyset$ we have that $^cO_i$ is finite and a finite union
        of finites is finite. We conclude that $O$ is open set.
    \end{enumerate}
    By (a), (b) and (c), $\T$ is a topology on $\Z$. 

    \item Exhibit an sequence of open sets $\{O_n\}_{n\in\mathbb{N}} \subset
    \T$ such that $\bigcap_{n \in \mathbb{N}} O_n$ is not an open
    set.

    Let $O_n = ~^c\{1, ..., n\}$. Thus $^c O_n = \{1,...,n\}$ is finite and
    $$^c\left(\bigcap_{n \in \mathbb{N}} O_n\right) = \bigcup_{n \in
    \mathbb{N}} ~^cO_n = \bigcup_{n \in
    \mathbb{N}} \{1,...,n\} = \mathbb{N},$$
    that is not finite. Therefore, this intersection is not an open set.     
    
\end{enumerate}

\noindent\linia

\begin{exercise}
    Let $x \in \R^n$, and $r > 0$. Let $y \in \B(x, r)$. Show that
    $$\B(y,r - ||x-y||) \subset \B(x,r)$$
\end{exercise}

Let $z \in \B(y, r - ||x-y||)$, so $||z - y|| < r - ||x - y|| \implies ||z-y||
+ ||x-y|| < r$. We can conclude
that, by the triangular inequality,  
$$||x - z|| \le ||x - y|| + ||z - y|| < r.$$ 
In that sense, $z \in \B(x,r)$ and $\B(y, r - ||x - y||) \subset \B(x,r)$.

\begin{remark}
    In the notes, the exercise is to prove $\B(y,||x-y||) \subset \B(x,r)$,
    however, this does not hold, because if we take $y$ next the border of
    $\B(x,r), ||x - y|| \approx r$ and $B(y,r -\epsilon) \not \subset B(x,r)$.
\end{remark}

\noindent\linia

\begin{exercise}
    Let $x, y \in \R^n$, and $r = ||x - y||$. Show that
    $$
    \B\left(\frac{x + y}{2}, \frac{r}{2}\right) \subset \B(x, r) \cap \B(y, r)
    $$
\end{exercise}

Denote $m = \dfrac{x+y}{2}$. Take $z \in \B\left(m, \dfrac{r}{2}\right)$.
Thus, using the triangular inequality,

$$||x - z|| \le ||x - m|| + ||m - z|| = \frac{1}{2}||x - y|| + ||m - z|| < r/2
+ r/2 = r$$
$$||y - z|| \le ||y - m|| + ||m - z|| = \frac{1}{2}||y - x|| + ||m - z|| < r/2
+ r/2 = r$$

So $z \in \B(x,r)$, $z \in \B(y, r)$ and $z \in \B(x,r) \cap \B(y, r)$.
Therefore $\B(m, \frac{r}{2}) \subset \B(x, r) \cap \B(y, r)$.

\noindent\linia

\begin{exercise}
    Show that the open balls $\B(x, r)$ of $R^n$ are open sets (with respect
    to the Euclidean topology).
\end{exercise}

We have to prove that for every $y \in \B(x, r)$, there exists $\epsilon > 0$ such
that $\B(y, \epsilon) \subset \B(x, r)$. Put $\epsilon = r - ||x - y||$. As
we have proved in exercise 3, $\B(y, \epsilon) \subset \B(x,r)$. 
So $\B(x,r)$ is open set. 

\noindent\linia

\begin{exercise}
    Consider $X = \R$ endowed with the Euclidean topology. Are the following
sets open? Are they closed?
\end{exercise}
\begin{enumerate}
    \item $[0,1]$. It's not open set because for every $\epsilon > 0, \B(0,
    \epsilon) = (-\epsilon, \epsilon) \not \subset [0,1]$. It's closed because
    $[0,1]^c = (-\infty, 0) \cup (1, \infty)$ is an union of two open sets, as
   we prove in item 3.  

    \item $[0,1)$. It's not open for the same reason as before. It's not
    closed because $B(1,\epsilon) = (1 - \epsilon, 1 + \epsilon) \not\subset
    (-\infty, 0) \cup [1, \infty]$. 

    \item $(-\infty,1)$. It's open because: take $x < 1$. Put $r = 1 - x$ and
    take $z \in \B(x, r)$. If $z > x$, $|x - z| < 1 - x \implies z < 1$. If $z
    < x$, it follows $z < 1$. It proves $z < 1$ and $(-\infty, 1)$ is open.
    It's not closed cause $\forall \epsilon > 0, \B(1,\epsilon) \not \subset
    [1, \infty)$. 
    
    \item the singletons. It's not open cause $\forall \epsilon > 0, x +
    \epsilon/2 \in \B(x,\epsilon)$. It's close cause $(-\infty,
    x)\cup(x,\infty)$ is union of open sets. 
    
    \item $\mathbb{Q}$. It's not open because for every open ball around a
    rational, there is irrationals, that is, for $x \in \mathbb{Q}$ and
    $\forall \epsilon > 0$, exists $y \in (\R - \mathbb{Q}) \cap
    \B(x,\epsilon)$. It's not closed for the same reason, for every
    irrational, there is rationals for every open ball. 
\end{enumerate}

\noindent\linia

\begin{exercise}
    A map is continuous if and only if the preimage of closed sets are closed sets.
\end{exercise}

First we shall prove that $f^{-1}(~^cA) = ~^c(f^{-1}(A))$. Let's prove the
double inclusion. Take $x \in f^{-1}(~^cA)$. So there exists $y \in ~^cA$ such
that $f(x) = y$. Suppose that $x \in f^{-1}(A)$. It implies the existence of
$z \in A$ such that $y = f(x) = z$, absurd. So $x \in ~^c(f^{-1}(A))$.

Now take $x \in ~^c(f^{-1}(A))$. Therefore, $\forall y \in A, f(x) \neq y$. In
that case, $f(x) \in ~^cA \implies x \in f^{-1}(~^cA)$. Then we have showed
the equality. 

\vspace{5mm}

Now let's prove the equivalence. Suppose $f$ is a continuos map and take a
closed set $F$. We shall prove that $f^{-1}(F)$ is closed. Well, $^c(f^{-1}(F))
= f^{-1}(~^cF)$ is open, because $~^cF$ is open, by the continuity. We
conclude that $f^{-1}(F)$ is closed. 

Suppose that for every closed set $F$, we have $f^{-1}(F)$ being closed. We
will use that $A$ is open if $^cA$ is closed. This is true because $^c(^c(A))
= A$. Take an open set $A$. $^c(f^{-1}(A)) = f^{-1}(~^cA)$
is closed, because $~^cA$ is. Thus $f^{-1}(A)$ is open and we have proved the
continuity of $f$. 


\section{Homeomorphisms}

\subsection{Important definitions}

\begin{definition}
    Let $(X, \T)$ and $(Y, \U)$ be two topological spaces, and $f : X
    \to Y$ a map. We say that $f$ is a homeomorphism if
    \begin{enumerate}
        \item $f$ is a bijection, 
        \item $f : X \to Y$ is continuos, 
        \item $f^{-1} : Y \to X$ is continuos.
    \end{enumerate}
    If there exists such a homeomorphism, we say that the two topological
    spaces are homeomorphic.
\end{definition}

\begin{definition}
    Let $(X, \T)$ be a topological space. We say that $X$ is connected if for
    every open sets $O, O' \in \T$ such that $O \cap O' = \emptyset$ (i.e.,
    they are disjoint), we have
    $$X = O \cup O' \implies O = \emptyset \text{ or } O' = \emptyset.$$
\end{definition}

\begin{definition}
    Let $(X, \T)$ be a topological space. Suppose that there exists a
    collection of $n$ \textbf{non-empty}, \textbf{disjoint} and
    \textbf{connected open sets} $(O_1, ..., O_n)$ such that
    $$
    \bigcup_{1 \le i \le n} O_i = X.
    $$
    Then we say that $X$ admits $n$ connected components.
\end{definition}

\begin{definition}
    Let ($X, \T)$ be a topological space, and $n \geq 0$. We say that it has
    dimension $n$ if the following is true: for every $x \in X$, there exists
    an open set $O$ such that $x \in O$, and a homeomorphism $O \to \R^n$.
\end{definition}

\subsection{Exercises}

\begin{exercise}
    Show that the topological spaces $\R^n$ and $\B(0, 1) \subset \R^n$ are homeomorphic.
\end{exercise}

Let $f: \B(0,1) \to \R^n$ be defined as $f(x) = \dfrac{x}{1 - ||x||}$. I
observe it's well defined because $||x|| < 1$. We shall prove $f$ is a
homeomorphism. 
\begin{enumerate}
    \item \textbf{Injective:} Take $x, y \in \B(0,1)$ and suppose that  
    $$
    \frac{x}{1 - ||x||} = \frac{y}{1 - ||y||}. 
    $$
    Applying the norm in both sides, we obtain the equation 
    $$||x||(1 - ||y|||) = ||y||(1 - ||x||) \implies ||x|| = ||y||.$$
    On the other side $x$ and $y$ points to the same direction, given that 
    $$
    y = \frac{1 - ||y||}{1 - ||x||}x = \alpha x, 
    $$
    with $\alpha = 1$ because of the same norm. We conclude $x = y$. 

    \item \textbf{Surjective:} Take $y \in \R^n$. We shall prove that there
    exists $x \in \B(0,1)$ such that $f(x) = y$, that is, 
    $$
    \frac{x}{1 - ||x||} = y
    $$
    Applying the norm we observe that if that is true, $||x|| = ||y|| -
    ||y||||x|| \implies ||x|| = \frac{||y||}{1 + ||y||}$. And $x = (1 -
    ||x||)y = \frac{1}{1 + ||y||}y$. We conclude that for every $y \in \R^n$,
    if we take $x = \frac{y}{1 + ||y||}$,
    $$
    f(x) = \frac{y/(1 + ||y||)}{1 - ||y||/(1 + ||y||)} = y
    $$

    \item \textbf{Continuity of f:} Consider an open set $A \subset \R^n$. Let
    $B = f^{-1}(A)$. We shall prove $B$ is open, that is, for every $x \in B$,
    exists $r > 0$ such that $\B(x, r) \subset B$. Take $x = f^{-1}(y) \in B$.
    Because $A$ is open, there is $\epsilon > 0$ such that $\B(y, \epsilon)
    \subset A$. Take $\delta$ such that 
    $$
    \frac{\delta}{1 - ||x|| - \delta}(1 + ||y||) < \epsilon
    $$
    and $z = f^{-1}(w) \in \B(x,\delta)$. 
    \begin{equation*}
        \begin{split}
            ||y - w|| &= \left|\left|\frac{x}{1 - ||x||} - \frac{z}{1 - ||z||}\right|\right| = \frac{1}{1 - ||x||}\left|\left|x - \frac{1 - ||x||}{1 - ||z||}z\right|\right| \\ 
            &= \frac{1}{1 - ||x||}\left|\left|x - z + z - \frac{1 - ||x||}{1 - ||z||}z\right|\right| \\
            &\le \frac{||x-z||}{1 - ||x||} + \frac{1}{1 - ||x||}\left(1 - \frac{1 - ||x||}{1 - ||z||}||z||\right) \\
            &= \frac{||x-z||}{1 - ||x||} + \frac{||z||}{1 - ||x||}\frac{||x|| - ||z||}{1 - ||z||} \\
            &\le \frac{1}{1 - ||x||}||x-z||(1 + ||w||) \\
            &\le \frac{1}{1 - ||x||}||x-z||(1 + ||y - w|| + ||y||) \\
            \implies ||y - w|| &\le \frac{||x-z||}{1 - ||x|| - ||x-z||}(1 + ||y||) \\ 
            &< \frac{\delta}{1 - ||x|| - \delta}(1 + ||y||) < \epsilon
        \end{split}
    \end{equation*}
    So $w \in \B(y, \epsilon) \subset A \implies z \in B$, what proves $B$ is
    open. It concludes the continuity of $f$. 

    \item \textbf{Continuity of $f^{-1}$: } The inverse is given by 
    $$
    f^{-1}(y) = \frac{y}{1 + ||y||}
    $$
    The demonstration is quite similar to the previous item, given that the
    only difference is the signal. 
\end{enumerate}

By items (1) - (4), we conclude $f$ is a homeomorphism and $\B(0,1) \simeq
\R^n$. 

\noindent\linia

\begin{exercise}
    Show that $\B(x, r)$ and $\B(y, s)$ are homeomorphic.
\end{exercise}

Consider the function $f: \B(0, 1) \to \B(c, r)$ given by $f(x) = r\cdot x + c$. Let's
prove $f$ is a homeomorphism. 
\begin{enumerate}
    \item \textbf{Injective:} If $x,y \in \B(0,1)$ and $rx + c = ry + c
    \implies x = y$, because $r > 0$ by 
    definition. So $f$ is injective. 
    
    \item \textbf{Surjective:}  Let $y \in \B(c, r)$ and $x = (y - c)/r$. So
    $||x|| = ||y - c||/r < 1$, by definition. So $x \in \B(0,1)$ and $f(x) =
    y$ what proves this function is surjective. 

    \item \textbf{Continuity of f:} Let $A \subset \B(c,r)$ open set and
    denote $B = f^{-1}(A)$. Take $x = f^{-1}(y) \in B$. We know there exists
    $\epsilon > 0$ such that $\B(y, \epsilon) \subset A$. Define $\delta = \epsilon/r$ and
    take $z = f^{-1}(w) \in \B(x,\delta)$. 
    $$
    ||y-w|| = ||rx + c - (rz + c)|| = r||x - z|| < r\delta = \epsilon 
    $$
    Therefore $w \in \B(y, \epsilon) \subset A \implies z \in B$. So
    $\B(x,\delta) \subset B$, what proves $B$ is open. This concludes the
    continuity of $f$. 
    
    \item \textbf{Continuity of $f^{-1}$:} The inverse is given by 
    $$
    f^{-1}(y) = \frac{y - c}{r}
    $$
    This function is continuos for the same argument as before. 
\end{enumerate}

By items (1) - (4), we conclude $f$ is a homeomorphism and $\B(0,1) \simeq
\B(c,r)$. Since this is a equivalence relation, we have that
$$
\B(0,1) \simeq \B(x,r) \text{ and } \B(0,1) \simeq \B(y, s) \text{ implies } \B(x,r) \simeq \B(y,s).
$$

\noindent\linia

\begin{exercise}
    Show that $\mathbb{S}(0, 1)$, the unit circle of $\R^2$, is homeomorphic
    to the ellipse 
    $$
    \mathcal{S}(a,b) = \left\{(x,y) \in \R^2,\left(\frac{x}{a}\right)^2 + \left(\frac{y}{b}\right)^2 = 1 \right\},
    $$
    for any $a, b > 0$. 
\end{exercise}

Consider the function $f : \mathbb{S}(0,1) \to \mathcal{S}(a,b)$ such that
$f(x,y) = (ax, by)$. Let's prove it is a homeomorphism. 

\begin{enumerate}
    \item \textbf{Injective:} Let $(x_1, y_1), (x_2, y_2) \in \mathbb{S}(0,1)$
    such that $(ax_1, by_1) = (ax_2, by_2)$. Since $a, b > 0$, we have $x_1 =
    x_2$ and $y_1 = y_2$. It proves $f$ is injective. 

    \item \textbf{Surjective:} Let $(z,w) \in \mathcal{S}(a,b)$ and $(x,y) =
    \left(\dfrac{z}{a}, \dfrac{w}{b}\right)$. It's clear that $f(x,y) = (z,w)$
    and $x^2 + y^2 = \frac{z^2}{a^2} + \frac{w^2}{b^2} = 1$, so $(x,y) \in
    \mathbb{S}(0,1)$. It proves $f$ is surjective. 
    
    \item \textbf{Continuity of f:} Let $A \subset \mathcal{S}(a,b)$ open set and denote
    $B = f^{-1}(A)$. Take $(x,y) = f^{-1}(z,w) \in B$. We know there exists
    $\epsilon > 0$ such that $\B((z,w), \epsilon) \subset A$. Put $\delta$ as
    defined below and 
    take $(x', y') = f^{-1}((z', w')) \in \B((x,y),\delta)$. Consider the norm
    1
    \begin{equation*}
        \begin{split}
            ||(z',w')-(z,w)||_1 &= ||(ax', by') - (ax, by)||_1 = ||\left(a(x' - x), b(y' - y)\right)||_1 \\ 
            &= a|x' - x| + b|y' - y|, \text{ define } c = \max\{a,b\}\\        
            &\le c(|x' - x| + |y' - y|) = c||(x' - x, y' - y)||_1
        \end{split}
    \end{equation*}

    By the equivalente of the norms, there exists constants $k_1, k_2$ such that 
    $$
    ||(z',w')-(z,w)|| \le k_1||(z',w')-(z,w)||_1 \le ck_1||(x' - x, y' - y)||_1 \le ck_1k_2||(x' - x, y' - y)|| 
    $$
    Then we need $\delta = \frac{\epsilon}{c k_1 k_2}$ in order to prove that
    $(z',w') \in \B((z,w), \epsilon) \subset A \implies (x',y') \in B$. So
    $\B((x,y),\delta) \subset B$, what proves $B$ is open. This concludes the
    continuity of $f$. 

    \item \textbf{Continuity of $f^{-1}$:} The inverse is given by 
    $$
    f^{-1}((z,w)) = (z/a, w/b)
    $$
    This function is continuos for the same argument as before. 
\end{enumerate}

By items (1) - (4), we conclude $f$ is a homeomorphism and $\mathbb{S}(0,1) \simeq
\mathcal{S}(a,b)$.

\noindent\linia

\begin{exercise}
    Show that $[0, 1)$ and $(0, 1)$ are not homeomorphic.
\end{exercise}

We shall prove by contradiction. Suppose these exists a homeomorphism $f :
[0,1) \to (0,1)$. Let $0 < z = f(0) < 1$ and define the following function
\begin{align*}
    g : (0,1) &\to (0,z) \cup (z,1) \\
    x &\mapsto g(x) = f(x)
\end{align*}

This function is well defined given that $z$ is not image of other point but
$0$. The function is injective because if $g(y) = g(x) \implies f(y) = f(x)
\implies x = y$, given that $f$ is injective. This function is also surjective
since $f$ is and $0 < w < 1$ and $w \neq z$, it's clear that $f(0) \neq w$. As
$g$ is an induced map of a continuos function, by Proposition 1.21 from the
notes, it's continuos and so is its inverse. We conclude $g$ is a
homeomorphism. 

Now I will prove that $(0,1)$ admits only 1 connected component, that is, it's
connected. Suppose it's not and there exists $O, O' \subset (0,1)$ open
disjoint sets such that $(0,1) = O \cup O'$ and none of them are
empty sets. Let $a \in O, b \in O'$ with $a < b$ without loss of generality.
Define $\alpha = \sup\{x \in \R : [a,x) \subset O\}$. It's well
defined because this set is not empty, given $O$ is open and $b$ is an upper
bound. Then $\alpha \leq b$. Suppose $\alpha \in O'$, then there exists $r >
0$ such that $(\alpha - r,\alpha + r) \subset O'$. We know that for every
$\epsilon > 0$, there exists $w \in (\alpha - \epsilon, \alpha]$ such that
$[a, w) \subset O$. That is a contradiction since there exists $w \in (\alpha
- r, \alpha)$ such that $[a, w) \subset O$. So $\alpha \in O \implies (\alpha
- r, \alpha + r) \subset O$, for some $r$. We infer that $[a,\alpha + r) \subset
O$, what is an absurd. Therefore $(0,1)$ is connected. 

For a similar argument, we prove that $(0,z)$ and $(z,1)$ are connected. This
implies that the union admits 2 connected components.

In that sense, we have a homeomorphism between a topological space with 1
connected component and other with 2 connected components, what is a
contradiction by Proposition 2.14 from the notes. We conclude that $[0,1)$ and
$(0,1)$ are not homeomorphic.

\section{Homotopies}

\subsection{Important definitions}

\begin{definition}
    Let $(X, \T)$ and $(Y, \U)$ be two topological spaces, and $f, g : X \to Y$
    two continuous maps. A homotopy between $f$ and $g$ is a map $F : X \times
    [0, 1] \to Y$ such that:
    \begin{enumerate}
        \item $F(\cdot, 0)$ is equal to $f$, 
        \item $F(\cdot, 1)$ is equal to $g$, 
        \item $F : X \times [0, 1] \to Y$ is continuous.
    \end{enumerate}
    If such a homotopy exists, we say that the maps f and g are homotopic.
\end{definition}

\begin{remark}
    Before asking for $F : X \times [0, 1] \to Y$ to be continuous, we have to
    give $X \times [0, 1]$ a topology. The topology we choose is the product
    topology. Consider the topological space $(X, \T)$, and endow $[0, 1]$
    with the subspace topology of $\R$, denoted $T_{|[0,1]}$. The product
    topology on $X \times [0, 1]$, denoted $T \otimes T_{|[0,1]}$, is defined
    as follows: a set $O \subset X \times [0, 1]$ is open if and only if it
    can be written as a union $U_{\alpha \in A} O_{\alpha} \times O_{\alpha}'$
    where every $O_{\alpha}$ is an open set of $X$ and $O_{\alpha}'$ is an open set of $[0, 1]$.
\end{remark}
 
\begin{definition}
    Let $(X, \T)$ and $(Y, \U)$ be two topological spaces. A homotopy
    equivalence between $X$ and $Y$ is a pair of continuous maps $f : X \to Y$
    and $g : Y \to X$ such that:
    \begin{enumerate}
        \item $g \circ f : X \to X$ is homotopic to the identity map $id: X \to X$,
        \item $f \circ g : Y \to Y$ is homotopic to the identity map $id: Y \to Y$,
    \end{enumerate}
    If such a homotopy equivalence exists, we say that $X$ and $Y$ are homotopy equivalent.   
\end{definition}

\begin{definition}
    Let $(X, \T)$ be a topological space and $Y \subset X$ a subset, endowed
    with the subspace topology $T_{|Y}$. A retraction is a continuous map $r :
    X \to Y$ such that $\forall y \in Y, r(y) = y$. 
    
    A deformation retraction is a homotopy $F : X \times [0, 1] \to Y$ between
    the identity map $id: X \to X$ and a retraction $r : X \to Y$.
\end{definition}

\subsection{Exercises}

\begin{exercise}
    Let $f : \R^n \to X$ be a continuous map. Then $f$ is homotopic to a constant map.
\end{exercise}

I must prove that there exists a homotopy between $f$ and a constant map.
Consider the function $F : \R^n \times [0,1] \to X$ defined as 
$$
F(x,t) = f(tx) 
$$
It's clear that $F(x,0) = f(0)$, for every $x \in \R^n$. So it's the constant
map $f(0)$. We also have that $F(x,1) = f(x), \forall x \in \R^n$. 
Moreover, let's prove $F$ is continuos. Denote $F' : \R^n \times \R \to X$ the
function $F'(x,t) = f(xt)$ and $g: \R^n \times \R \to \R^n$ the function
$g(x,t) = xt$. So $F' = f \circ g$. 

Let's prove $g$ is a continuous function. As we are dealing with a real-valued
function, by Proposition 1.19 from the notes, I can use the $\epsilon-\delta$
proof. Let $(x,t) \in \R^{n+1}$ and $\epsilon > 0$. In the proof I use the
norm 1, without loss of generality because of the equivalence of norms in
$\R^n$. Put $\delta = \min\{1, \frac{\epsilon}{\max\{||x||, |t| + 1\}} \}$ and suppose $||(x,t) - (x',t')|| = ||x - x'|| + |t -
t'| < \delta$. So,
\begin{equation*}
    \begin{split}
        ||xt - x't'|| &= ||xt - xt' + xt' - x't'|| \\
        &\le |t - t'|||x|| + |t'|||x - x'|| \\
        &\le |t - t'|||x|| + (|t| + \delta)||x - x'|| \\
        &< \max\{||x||, |t| + \delta\}\delta \\
        &\le \max\{||x||, |t| + 1\}\delta \le \epsilon
    \end{split}
\end{equation*} 

By this, $g$ is a continuos function. Since $f$ is also continuos, the
composition $F'$ is also continuos, by Proposition 1.18. By Proposition 1.21,
when we endow $F'$ in $\R^n \times [0,1]$, we obtain a continuos function,
that is $F$ is continuos. Then we conclude that $f$ is homotopic to a constant
function. 

\noindent\linia

\begin{exercise}
    Show that every map $f : \mathbb{S}_1 \to \mathbb{S}_2$ is homotopic to a
    constant map, where the unit sphere $\mathbb{S}_{n-1} = \{x \in \R^n,
    ||x|| = 1\}$. 
\end{exercise}

\begin{remark}
    I will suppose $f$ is continuous, otherwise I think it's not possible to
    prove there is a homotopy.     
\end{remark}

\noindent\linia

\begin{exercise}
    Show that being homotopic is a transitive relation between maps: for every
    triplet of maps $f, g, h: X \to Y$, if $f$, $g$ are homotopic and $g$, $h$
    are homotopic, then $f$, $h$ are homotopic.
\end{exercise}

We shall prove there exist a homotopy $H$ between $f$ and $h$. By assumption,
there exists a homotopy $F$ between $f$ and $g$ and a homotopy $G$ between $g$
and $h$. Define $H: X \times [0,1] \to Y$ such that 
$$
H(x,t) = \begin{cases}
    F(x,2t), &0 \le t \le 1/2 \\
    G(x,2t - 1), &1/2 < t \le 1   
\end{cases}
$$
that is, $H$ behaves as $F$ until it reaches a half. When that occurs,
$H(x,1/2) = F(x,1) = g(x) = G(x, 0)$. After that, $H$ follows $G$ until the
end of the interval. So, it's clear that $H(x,0) = F(x,0) = f(x), \forall x
\in X$ and $H(x,1) = G(x,1) = h(x), \forall x \in X$. Moreover, since $F$ and
$G$ are continuos and in the point $t = 1/2$, both functions agree, $H$ is
continuos and, therefore, $f$ and $h$ are homotopic. 

\noindent\linia

\begin{exercise}
    Show that being homotopy equivalent is an equivalence relation (reflexive,
    symmetric and transitive).
\end{exercise}

\begin{enumerate}
    \item (\textit{reflexive}): Consider the identity map $id: X \to X$,
    that is continuos.
    We shall prove that this function is homotopic to itself. Consider $F : X
    \times [0,1] \to X$ given by $F(x,t) = x$ for every $x$ and $t$. It's
    clear this is a homotopy because $F(x,0) = F(x,1) = x$ and it's continuos. Moreover
    $id \circ id = id$ by definition of identity. Therefore, there exists a
    homotopy equivalence between $id$ and itself. We conclude $X \approx X$.  

    \item (\textit{symmetric}): Suppose $X \approx Y$. So, there exists
    continuos functions $f: X \to Y$ and $g: Y \to X$ that form a homotopy
    equivalence. This means that $g: Y \to X$ and $f: X \to Y$ are a homotopy
    equivalence as well. So $Y \approx X$.

    \item (\textit{transitive}): Suppose $X \approx Y$, and let $f_1 : X \to
    Y$ and $g_1: Y \to X$
    form a homotopy equivalence. Also suppose $Y \approx Z$ and
    let $f_2: Y \to Z$ and $g_2: Z \to Y$ form a homotopy equivalence.
    Define $f_3 = f_2 \circ f_1$ and $g_3 = g_1 \circ g_2$. Let's proof this
    is a homotopy equivalence. Both functions are continuos given that they
    are a composition of continuos functions. 

    \begin{enumerate}
        \item $g_3 \circ f_3 = g_1 \circ g_2 \circ f_2 \circ f_1$ is homotopic
        to $id: X \to X$.

        Let $F_1$ be a homotopy between $g_1 \circ f_1$ and $id$ and $F_2$ a
        homotopy between $g_2 \circ f_2$ and $id$. Define 
        $$
        F_3(x,t) = \begin{cases}
            g_1 \circ F_2(\cdot,2t)\circ f_1(x), &0 \le t \le 1/2 \\
            F_1(x, 2t - 1), &1/2 < t \le 1
        \end{cases}
        $$
        So $F_3(x,0) = g_1(F_2(f_1(x),0)) = g_1(g_2(f_2(f_1(x)))) = g_3 \circ
        f_3(x)$, for every $x$ and 
        $F_3(x,1) = F_1(x,1) = x$, for every $x$. When $t = 1/2$, 
        $$F_3(x, 1/2) = g_1(F_2(f_1(x), 1)) = g_1(f_1(x)) = F_1(x,0)$$
        By this equality and the fact that composition of continuos
        functions is a continuos map, we conclude that $F_3$ is continuos.
        This implies that $g_3\circ f_3$ is homotopic to the identity.

        \item $f_3 \circ g_3 = f_1 \circ f_2 \circ g_2 \circ g_1$ is homotopic
        to $id: Z \to Z$.

       This follows a quite similar demonstration and can be omitted. 

    \end{enumerate}

    By the points above $f_3$ and $g_3$ is a homotopy equivalence what proves
    $X \approx Z$.    
\end{enumerate}

Consequently, homotopy equivalence is a equivalence relation.

\noindent\linia

\begin{exercise}
    Classify the letters of the alphabet into homotopy equivalence classes.
\end{exercise}

I will consider the upper case alphabet and each letter will be considered as
a topological space (a subset from $\R^2$), for example the letter $O$ is
homotopy equivalent to a circle, while $L$ is to an interval, or equivalently,
a point. Observe that most of the letters are equivalent to a point, because
we can think in a continuos reduction. When we have a hole, such as $A, D, R, O,
P, Q$, this continuity is impossible since we'll have a point break. $B$ is a
special case because we can't deform into a point without breaking points and
also we cannot join the holes in one. So there is three classes, given by its
representatives

\begin{enumerate}
    \item O
    \item B
    \item I
\end{enumerate}

\begin{center}
    \begin{tabular}{c c c c c c c c c c c c c}
     A & B & C & D & E & F & G & H & I & J & K & L & M   \\ 
     1 & 2 & 3 & 1 & 3 & 3 & 3 & 3 & 3 & 3 & 3 & 3 & 3 
    \end{tabular}
    \end{center}

\begin{center}
    \begin{tabular}{c c c c c c c c c c c c c}
     N & O & P & Q & R & S & T & U & V & W & X & Y & Z  \\ 
     3 & 1 & 1 & 1 & 1 & 3 & 3 & 3 & 3 & 3 & 3 & 3 & 3     
    \end{tabular}
    \end{center}

\end{document}