\documentclass[a4paper,11pt]{article}

\usepackage[T1]{fontenc}
\usepackage[utf8]{inputenc}
\usepackage{amsmath,amssymb,amsthm,textcomp}

\usepackage[
pdftitle={Topological Exercises}, 
pdfauthor={Lucas Moschen, Fundação Getulio Vargas},
colorlinks=true,linkcolor=blue,urlcolor=blue,citecolor=blue,bookmarks=true,
bookmarksopenlevel=2]{hyperref}
\usepackage{enumerate}
\usepackage{multicol}
\usepackage{setspace}
\onehalfspacing

\usepackage{graphicx}
\usepackage{xcolor}
\usepackage{tikz}
\usepackage{geometry}
\geometry{left=25mm,right=25mm,%
bindingoffset=0mm, top=20mm,bottom=20mm}

\newcommand{\linia}{\rule{\linewidth}{0.5pt}}

% custom theorems if needed
\newtheoremstyle{mytheor}
    {1ex}{1ex}{\itshape}{0pt}{\scshape}{.}{1ex}
    {{\thmname{#1 }}{\thmnumber{#2}}{\thmnote{ (#3)}}}

\theoremstyle{mytheor}
\newtheorem{definition}{Definition}[subsection]

\theoremstyle{mytheor}
\newtheorem{exercise}{Exercise}[subsection]

\theoremstyle{remark}
\newtheorem*{remark}{Remark}

\newcommand{\T}{\mathcal{T}}
\newcommand{\B}{\mathcal{B}}
\newcommand{\Z}{\mathbb{Z}}
\newcommand{\R}{\mathbb{R}}


% my own titles
\makeatletter
\renewcommand{\maketitle}{
    \begin{center}
        \vspace{2ex}
        {\huge \textsc{\@title}}
        \vspace{1ex}
        \\
        \linia\\
        \@author \hfill \@date
        \vspace{4ex}
    \end{center}
}
\makeatother

%%%----------%%%----------%%%----------%%%----------%%%

\title{Topological Data Analysis - Exercises}

\author{Lucas Moschen, EMAp/FGV}

\date{\today}

\begin{document}

\maketitle

\section{General topology}

\subsection{Important definitions}

\begin{definition}
    A topological space is a pair $(X, \T)$ where $X$ is a set and $\T$ is a
    collection of subsets of $X$ such that:
    \begin{enumerate}
        \item $\emptyset \in \T, X \in \T$. 
        \item for every infinite collection $\{O_{\alpha}\}_{\alpha \in A}\subset \T$, we have $\bigcup_{\alpha \in A} O_{\alpha} \in \T$.
        \item for every finite collection $\{O_{i}\}_{1 \le i \le n}
        \subset \T$, we have $\bigcap_{1 \le i \le n} O_{i} \in \T$.
    \end{enumerate}   
\end{definition}

\begin{definition}
    Let $x \in \R^n$ and $r > 0$. The open ball of center $x$ and radius $r$,
    denoted $\B(x, r)$, is defined as: $\B(x, r) = \{y \in \R^n, ||x - y|| < r\}$.
\end{definition}

\begin{definition}
    Let $A \subset \R$ and $x \in A$. We say that $A$ is open
    around $x$ if there exists $r > 0$ such that $\B(x, r) \subset A$. We say
    that $A$ is open if for every $x \in A$, $A$ is open around $x$.
\end{definition}

\begin{definition}
    Let $(X, \T)$ be a topological space, and $Y \subset X$. We define the subspace topology on $Y$ as the following set:
    $$T_{|Y} = \{O \cap Y, O \in \T\}$$
\end{definition}

\begin{definition}
    Let $f : X \to Y$ be a map. We say that $f$ is continuous if for every
    $O \in U$, the preimage $f^{-1}(O) = \{x \in X, f(x) \in O\}$ is in $\T$.
\end{definition}

\subsection{Exercises}

\begin{exercise}
    Let $X = \{0, 1, 2\}$ be a set with three elements. What are the  different topologies that X admits?
\end{exercise}

As we know every Topology contains $\emptyset$ and $\{0,1,2\}$, so we can
disconsider when writing the topologies, that is, all below contain these
subsets. 

\begin{itemize}
    \item (2) Basic: $\{\emptyset, \{0,1,2\}\}$ - $\mathcal{P}(\{0,1,2\})$.  
    \item (8) With $\{0\}$: $\{\{0\}\} - \{\{0\}, \{0,1\}\} - \{\{0\}, \{1,2\}\} -
    \{\{0\}, \{0,2\}\} - \{\{0\}, \{0,2\}, \{0,1\}\}$ \\ 
    $\{\{0\}, \{2\}, \{0,2\}\}- \{\{0\}, \{2\}, \{0,2\}, \{1,2\}\} -
    \{\{0\}, \{2\}, \{0,2\}, \{0,1\}\}$

    \item (8) With $\{1\}$: $\{\{1\}\} - \{\{1\}, \{0,1\}\} - \{\{1\}, \{1,2\}\} -
    \{\{1\}, \{0,2\}\} - \{\{1\}, \{1,2\}, \{0,1\}\}$ \\
    $\{\{0\}, \{1\}, \{0,1\}\} - \{\{0\}, \{1\}, \{0,1\}, \{1,2\}\} -  \{\{0\}, \{1\}, \{0,1\}, \{0,2\}\}$

    \item (8) With $\{2\}$: $\{\{2\}\} - \{\{2\}, \{0,1\}\} - \{\{2\}, \{1,2\}\} -
    \{\{2\}, \{0,2\}\} - \{\{2\}, \{0,2\}, \{1,2\}\}$ \\
    $\{\{1\}, \{2\}, \{1,2\}\} - \{\{1\}, \{2\}, \{1,2\}, \{0,1\}\} - \{\{1\},
    \{2\}, \{1,2\}, \{0,2\}\}$ 
    
    \item (3) No singleton: $\{\{0,1\}\} - \{\{1,2\}\} - \{\{0,2\}\} $
\end{itemize}

\noindent\linia

\begin{exercise}
    Let $\Z$ be the set of integers. Consider the \textit{cofinite
    topology} $\T$ on $\Z$, defined as follows: a subset $O \subset
    \Z$ is an open set if and only if $O = \emptyset$ or $^c O$ is finite. Here, $^cO = \{x \in \Z, x \not\in O\}$ represents the complementary of $O$ in $\Z$
\end{exercise}

\begin{enumerate}
    \item Show that $\T$ is a topology on $\Z$.

    Let's verify the three axioms: 

    \begin{enumerate}
        \item $\emptyset$ is an open set by definition and $\Z$ is open set
        because $^c\Z = \emptyset$ is finite. 

        \item Let $\{O_{\alpha}\}_{\alpha \in A}\subset \T$. So 
        $^cO = ~^c\left(\bigcup_{\alpha \in A} O_{\alpha}\right) = \bigcap_{\alpha
        \in A} ~^cO_{\alpha} \implies ^c O \subset ~^cO_{\alpha}, \forall
        \alpha \in A$. If $\forall \alpha, O_{\alpha} = \emptyset$, then $^cO =
        ~^c\emptyset \implies O = \emptyset$   and $O$ is open. On the other
        hand, if there exists $\alpha \in A$ such that $O_{\alpha} \neq
        \emptyset$ we have $^c O_{\alpha}$ being finite, so is $^c O$, given
        the inclusion. We conclude $O$ is open set. 
        
        \item Let $\{O_{i}\}_{1 \le i \le n} \subset \T$. So 
        $^cO = ~^c\left(\bigcap_{1 \le i \le n} O_i\right) = \bigcup_{1 \le i
        \le n} ~^cO_i$. If $O_i = \emptyset$ for some $1 \le i \le n$, $O =
        \emptyset$ because of the intersection. Alternatively, if $\forall i,
        O_i \neq \emptyset$ we have that $^cO_i$ is finite and a finite union
        of finites is finite. We conclude that $O$ is open set.
    \end{enumerate}
    By (a), (b) and (c), $\T$ is a topology on $\Z$. 

    \item Exhibit an sequence of open sets $\{O_n\}_{n\in\mathbb{N}} \subset
    \T$ such that $\bigcap_{n \in \mathbb{N}} O_n$ is not an open
    set.

    Let $O_n = ~^c\{1, ..., n\}$. Thus $^c O_n = \{1,...,n\}$ is finite and
    $$^c\left(\bigcap_{n \in \mathbb{N}} O_n\right) = \bigcup_{n \in
    \mathbb{N}} ~^cO_n = \bigcup_{n \in
    \mathbb{N}} \{1,...,n\} = \mathbb{N},$$
    that is not finite. Therefore, this intersection is not an open set.     
    
\end{enumerate}

\noindent\linia

\begin{exercise}
    Let $x \in \R^n$, and $r > 0$. Let $y \in \B(x, r)$. Show that
    $$\B(y,r - ||x-y||) \subset \B(x,r)$$
\end{exercise}

Let $z \in \B(y, r - ||x-y||)$, so $||z - y|| < r - ||x - y|| \implies ||z-y||
+ ||x-y|| < r$. We can conclude
that, by the triangular inequality,  
$$||x - z|| \le ||x - y|| + ||z - y|| < r.$$ 
In that sense, $z \in \B(x,r)$ and $\B(y, r - ||x - y||) \subset \B(x,r)$.

\begin{remark}
    In the notes, the exercise is to prove $\B(y,||x-y||) \subset \B(x,r)$,
    however, this does not hold, because if we take $y$ next the border of
    $\B(x,r), ||x - y|| \approx r$ and $B(y,r -\epsilon) \not \subset B(x,r)$.
\end{remark}

\noindent\linia

\begin{exercise}
    Let $x, y \in \R^n$, and $r = ||x - y||$. Show that
    $$
    \B\left(\frac{x + y}{2}, \frac{r}{2}\right) \subset \B(x, r) \cap \B(y, r)
    $$
\end{exercise}

Define $m = \dfrac{x+y}{2}$. Take $z \in \B\left(m, \dfrac{r}{2}\right)$.
Thus, using the triangular inequality,

$$||x - z|| \le ||x - m|| + ||m - z|| = \frac{1}{2}||x - y|| + ||m - z|| < r/2
+ r/2 = r$$
$$||y - z|| \le ||y - m|| + ||m - z|| = \frac{1}{2}||y - x|| + ||m - z|| < r/2
+ r/2 = r$$

So $z \in \B(x,r)$, $z \in \B(y, r)$ and $z \in \B(x,r) \cap \B(y, r)$.
Therefore $\B(m, \frac{r}{2}) \subset \B(x, r) \cap \B(y, r)$.

\noindent\linia

\begin{exercise}
    Show that the open balls $\B(x, r)$ of $R^n$ are open sets (with respect
    to the Euclidean topology).
\end{exercise}

We have to prove that for every $y \in \B(x, r)$, there exists $\epsilon > 0$ such
that $\B(y, \epsilon) \subset \B(x, r)$. Put $\epsilon = r - ||x - y||$. As
we have proved in exercise 3, $\B(y, \epsilon) \subset \B(x,r)$. 
So $\B(x,r)$ is open set. 

\noindent\linia

\begin{exercise}
    Consider $X = \R$ endowed with the Euclidean topology. Are the following
sets open? Are they closed?
\end{exercise}
\begin{enumerate}
    \item $[0,1]$. It's not open set because for every $\epsilon > 0, \B(0,
    \epsilon) = (-\epsilon, \epsilon) \not \subset [0,1]$. It's closed because
    $[0,1]^c = (-\infty, 0) \cup (1, \infty)$ is an union of two open sets, as
   we prove in item 3.  

    \item $[0,1)$. It's not open for the same reason as before. It's not
    closed because $B(1,\epsilon) = (1 - \epsilon, 1 + \epsilon) \not\subset
    (-\infty, 0) \cup [1, \infty]$. 

    \item $(-\infty,1)$. It's open because: take $x < 1$. Put $r = 1 - x$ and
    take $z \in \B(x, r)$. If $z > x$, $|x - z| < 1 - x \implies z < 1$. If $z
    < x$, it follows $z < 1$. It proves $z < 1$ and $(-\infty, 1)$ is open.
    It's not closed cause $\forall \epsilon > 0, \B(1,\epsilon) \not \subset
    [1, \infty)$. 
    
    \item the singletons. It's not open cause $\forall \epsilon > 0, x +
    \epsilon/2 \in \B(x,\epsilon)$. It's close cause $(-\infty,
    x)\cup(x,\infty)$ is union of open sets. 
    
    \item $\mathbb{Q}$. It's not open because for every open ball around a
    rational, there is irrationals, that is, for $x \in \mathbb{Q}$ and
    $\forall \epsilon > 0$, exists $y \in (\R - \mathbb{Q}) \cap
    \B(x,\epsilon)$. It's not closed for the same reason, for every
    irrational, there is rationals for every open ball. 
\end{enumerate}

\noindent\linia

\begin{exercise}
    A map is continuous if and only if the preimage of closed sets are closed sets.
\end{exercise}

First we shall prove that $f^{-1}(~^cA) = ~^c(f^{-1}(A))$. Let's prove the
double inclusion. Take $x \in f^{-1}(~^cA)$. So there exists $y \in ~^cA$ such
that $f(x) = y$. Suppose that $x \in f^{-1}(A)$. It implies the existence of
$z \in A$ such that $y = f(x) = z$, absurd. So $x \in ~^c(f^{-1}(A))$.

Now take $x \in ~^c(f^{-1}(A))$. Therefore, $\forall y \in A, f(x) \neq y$. In
that case, $f(x) \in ~^cA \implies x \in f^{-1}(~^cA)$. Then we have showed
the equality. 

\vspace{5mm}

Now let's prove the equivalence. Suppose $f$ is a continuos map and take a
closed set $F$. We shall prove that $f^{-1}(F)$ is closed. Well, $^c(f^{-1}(F))
= f^{-1}(~^cF)$ is open, because $~^cF$ is open, by the continuity. We
conclude that $f^{-1}(F)$ is closed. 

Suppose that for every closed set $F$, we have $f^{-1}(F)$ being closed. We
will use that $A$ is open if $^cA$ is closed. This is true because $^c(^c(A))
= A$. Take an open set $A$. $^c(f^{-1}(A)) = f^{-1}(~^cA)$
is closed, because $~^cA$ is. Thus $f^{-1}(A)$ is open and we have proved the
continuity of $f$. 


\section{Homeomorphisms}

\subsection{Important definitions}

\begin{definition}
    Let $(X, \T)$ and $(Y, \mathcal{U})$ be two topological spaces, and $f : X
    \to Y$ a map. We say that $f$ is a homeomorphism if
    \begin{enumerate}
        \item $f$ is a bijection, 
        \item $f : X \to Y$ is continuos, 
        \item $f^{-1} : Y \to X$ is continuos.
    \end{enumerate}
    If there exists such a homeomorphism, we say that the two topological
    spaces are homeomorphic.
\end{definition}

\subsection{Exercises}

\end{document}


