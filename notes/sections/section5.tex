\subsection{Important Definitions}

\begin{definition}
    Let $K$ be a simplicial complex. For any $n \ge 0$, define the sets
$$K_n = \{\sigma \in K, dim(\sigma) \le n\}$$
$$K_{(n)} = \{\sigma \in K, dim(\sigma) = n\}.$$
The first set is a simplicial complex, called the n-skeleton of $K$.
\end{definition}

\begin{definition}
    Let $n \ge 0$. The n-chains of $K$ is the set $C_n(K)$ whose elements are
    the formal sums
    $$\sum_{\sigma  \in K_{(n)}} \epsilon_{\sigma}\cdot \sigma \text{ where }
    \forall \sigma \in K_{(n)}, \epsilon_{\sigma} \in \Z/2\Z.$$
    We can give $C_n(K)$ a group structure via
    $$
    \sum_{\sigma \in K_{(n)}} \epsilon_{\sigma} \cdot \sigma + \sum_{\sigma \in K_{(n)}} \eta_{\sigma} \cdot \sigma = \sum_{\sigma \in K_{(n)}} (\epsilon_{\sigma} + \eta_{\sigma}) \cdot \sigma.$$
\end{definition}

\begin{definition}
    Let $n \ge 1$, and $\sigma \in K_{(n)}$ a simplex of dimension $n$. We
    define its boundary as the following element of $C_{n-1}(K)$:
    $$
    \partial_n \sigma = \sum_{\tau \subset \sigma, |\tau| = |\sigma| - 1} \tau
    $$ and 
    $$
    \partial_n \sum_{\sigma \in K_{(n)}} \epsilon_{\sigma} \cdot \sigma  =  \sum_{\sigma \in K_{(n)}} \epsilon_{\sigma} \cdot \partial_n \sigma 
    $$
\end{definition}

\begin{definition}
    We define
    \begin{enumerate}
        \item The n-cycles: $Z_n(K) = Ker(\partial_n)$,
        \item The n-boundaries: $B_n(K) = Im(\partial_{n+1}).$
    \end{enumerate}
    We say that two chains $c, c' \in C_n(K)$ are homologous if there exists
    $b \in B_n(K)$ such that $c = c' + b$.
\end{definition}

\begin{definition}
    The $n^{th}$ homology group of $K$ is $H_n(K) = Z_n(K)/B_n(K)$.
\end{definition}

\begin{definition}
    Let $K$ be a simplicial complex and $n \ge 0$. Its $n^{th}$ Betti number
    is the integer $\beta_n(K) = dim~ H_n(K)$.
\end{definition}

\begin{definition}
    The homology groups of a topological space are the homology groups of any triangulation of it. We define their Betti numbers similarly.
\end{definition}

\subsection{Exercises}

\begin{exercise}
    Let $V$ be a $\Z/2\Z$-vector space, and $W$ a linear subspace. Prove that
    $$\text{dim } V /W = \text{dim } V - \text{dim } W  .$$
\end{exercise}

\begin{proof}

Suppose $V$ is finite-dimensional, such that $dim ~V = n + m$ and $dim ~W =
m$. As we have a finite basis and a finite field, we only have finite
number of combinations from the vector of this basis. In special $V$ is finite. By Proposition 5.2,
$V$ has cardinal $2^{m+n}$ and $W$ has cardinal $2^m$. Let's prove that $V/W$
has cardinal $2^n$. We know the quotient space has finite dimension $k$, because
it's finite. So it has $2^k$ elements. Let $V/W = \{[v_1], ..., [v_{2^k}]\}$ where
$v_i$ represents an equivalence class. We know for each $v \in V$ there
exists $i$ such that $v \in v_i + W$ and there is $w \in W$ with $x =
v_i + w$. As each class is disjoint, we first choose one of the $2^k$ classes
$v_i$. After we pick out one element of this class. We have $2^m$ elements in
$W$. Let's see $|v_i + W| = 2^m$. Take $w, w' \in W$ and  suppose $v_i + w =
v_i + w'$. Summing $v_i$ in both sides we obtaing that $w = w'$. So we have
$2^k 2^m$ ways
of forming elements os $V$. We conclude $k$ must be $n$, what finishes our proof. 

\begin{remark}
    Consider the following proof slightly more general. 
\end{remark}

Suppose $V$ is finite-dimensional and let $\{w_1, ..., w_m\}$ be a basis of
$W$. So we can extend this to a basis on $V$, namely, $\{w_1, ..., w_m, v_1,
..., v_{n}\}$, where, $\text{dim } V = m + n$. Consider the set $\{v_1 + W, ...,
v_n + W\}.$ First, let's prove it's free.
$$
0 + W = \sum_{j=1}^n \lambda_j(v_j + W) = \left(\sum_{j=1}^n \lambda_j v_j \right) + W, 
$$
So $\left(\sum_{j=1}^n \lambda_j v_j \right) \in W$. But it implies it can be
written as a linear combination of $\{w_1, ..., w_m\}$, what contradicts the
fact that $\{w_1, ..., w_m, v_1,
..., v_{n}\}$ is linear independent. 

Now take $v + W \in V/W$, where $v = \sum_{j=1}^m \lambda_j w_j + \sum_{j=1}^n
\lambda_{j+m} v_j$. So 
\begin{equation*}
    \begin{split}
        v + W &= \left[\sum_{j=1}^m \lambda_j w_j + \sum_{j=1}^n
        \lambda_{j+m} v_j\right] + W \\
        &= \sum_{j=1}^m \lambda_j (w_j + W) + \sum_{j=1}^n
        \lambda_{j+m} (v_j+ W) \\
        &= \sum_{j=1}^n
        \lambda_{j+m} (v_j+ W) 
    \end{split}
\end{equation*}

We conclude $\{v_1,
..., v_{n}\}$ is a basis what implies $\text{dim } V/W = n = m + n - m$. 

\end{proof}

\noindent\linia

\begin{exercise}
    Let $(G, +)$ be a group, potentially non-commutative. Prove that
    $$\forall g \in G, g + g = 0 \implies G \text{ is commutative.}$$
\end{exercise}

\begin{proof}

Let $u,v \in G$. So $u + v + v = u + (v + v) = u + 0 = u$. With that in mind,
we see that 
$$
u + v + v + u = 0 \implies (u + v) + (v + u) = 0 
$$
We know $v + u$ and $u + v$ are elements of $G$. So we can add to each side
and obtain 
$$
(u + v) + (v + u) + (v + u) = (v + u) \implies u + v = v + u
$$
As $u$ and $v$ are arbitrary, $G$ is commutative. 

\end{proof}

\noindent\linia

\begin{exercise}
    Compute the Betti numbers $\beta_0(K), \beta_1(K)$ and $\beta_2(K)$ of the
    following simplicial complex:
    $$
    K = \{[0], [1], [2], [3], [0, 1], [1, 2], [2, 3], [3, 0]\}.
    $$
\end{exercise}

\begin{proof}

$Z_0(K) = C_0(K)$

$B_0(K) = \{[0] + [1], [1] + [2], [2] + [3], [3] + [0], [0] + [2], [1] + [3],
[0] + [1] + [2] + [3], 0\}$

As $B_0$ have $2^3$ elements and $Z_0$ has $2^4$, we deduce that 
$$\beta_0(K) = dim ~Z_0(K) - dim ~B_0(K) = 4 - 3 = 1$$

$Z_1(K) = \{[0,1] + [1,2] + [2,3] + [3,0], 0\}$

$B_1(K) = \{0\}$

Nesse caso, observamos que, utilizando a mesma ideia do ponto anterior 
$$\beta_1(K) = 1 - 0 = 1.$$
$$
K = \{[0], [1], [2], [3], [0, 1], [1, 2], [2, 3], [3, 0]\}.
$$
$Z_2(K) = \{0\}$

$B_2(K) = \{0\}$

Portanto $\beta_2(K) = 0$. 

\end{proof}

\noindent\linia

\begin{exercise}
    Compute the Betti numbers $\beta_0(K), \beta_1(K)$ and $\beta_2(K)$ of the
    following simplicial complex:
    $$
    K = \{[0], [1], [2], [3], [0, 1], [1, 2], [2, 3], [3, 0], [0, 2], [1, 3], [0, 1, 2], [0, 1, 3], [0, 2, 3], [1, 2, 3]\}.
    $$
\end{exercise}

\begin{proof}

$Z_0(K) = C_0(K)$

$B_0(K) = \{[0] + [1], [1] + [2], [2] + [3], [3] + [0], [0] + [2], [1] + [3],
[0] + [1] + [2] + [3], 0\}$

As $B_0$ have $2^3$ elements and $Z_0$ has $2^4$, we deduce that 
$$\beta_0(K) = dim ~Z_0(K) - dim ~B_0(K) = 4 - 3 = 1$$

\begin{multline}
    Z_1(K) = \{[0,1] + [1,2] + [0,2]; [0,1] + [0,3] + [1,3]; [0,2] + [2,3] +
    [0,3]; \\ 
    [1,2] + [2,3] + [1,3]; [0,1] + [1,2] + [2,3] + [0,3]; [0,1] + [1,3] + [2,3] + [2,0]; \\
    [0,2] + [1,2] + [1,3] + [0,3]; 0
    \}    
\end{multline}
\begin{multline}
    B_1(K) = \{[0,1] + [1,2] + [0,2]; [0,1] + [0,3] + [1,3]; [0,2] + [2,3] +
    [0,3]; \\ 
    [1,2] + [2,3] + [1,3]; 
    [0,1] + [1,2] + [2,3] + [0,3]; [0,1] + [1,3] + [2,3] + [0,2]; \\
    [0,2] + [1,2] + [1,3] + [0,3]; 0
    \}    
\end{multline}

Observe que os conjuntos são iguais e, portanto, 
$$\beta_1(K) = 0.$$

$Z_2(K) = \{[0,1,2] +[0,1,3] + [0,2,3] + [1,2,3], 0\}$

$B_2(K) = \{0\}$

Portanto $\beta_2(K) = 1$. 

\end{proof}