\subsection{Important definitions}

\begin{definition}
    Let $X \subset \R^n$ and $i \ge 0$. The $i^{th}$ Betti curve of $X$ is the
    map
    \begin{align*}
        \beta_i(t) : &\R^+ \to \mathbb{N} \\
        &t \mapsto \beta_i(X^t)     
    \end{align*}
    As a consequence of the nerve theorem, the map $t \to \beta_i(t)$ is equal
    to $t \mapsto \beta_i(Cech^t(X)).$ In practice, we may use the following
    map, called the $i^{th}$ Betti curve of the Rips complex of $X$ 
    \begin{align*}
        \beta_i^{Rips}(t) : &\R^+ \to \mathbb{N} \\
        &t \mapsto \beta_i(Rips^t(X))     
    \end{align*}
\end{definition}


\subsection{Exercises}

\begin{exercise}
    Show that $t \mapsto \beta_0(t)$ is non-increasing. Show that $t \mapsto
    \beta_0^{Rips}(t)$ is also non-increasing.
\end{exercise}

\begin{proof}


In order to prove the first statement, I shall find for each $t$ the value
$\beta_i(X^t)$, that is, we need to find $\beta_i(K)$, where $K$ is a
triangulation of $X^t$. Actually, I'll the weak version, where $|K|$ must be
homotopy equivalent to $X^t$. In particular, a consider the $Cech^t(X)$. Take
$s < t$ and consider $\mathcal{N}(\mathcal{V}^s), \mathcal{N}(\mathcal{V}^t)$
the nerves respectively. If $\{x_1, ..., x_m\}$ is m-simplex of
$\mathcal{N}(\mathcal{V}^s)$, we have that $\bigcap_{k=1}^m \bar{\B}(x_k, s)
\neq \emptyset$. For each $k \in [[1,m]], \bar{\B}(x_k, s) \subset
\bar{\B}(x_k, t)$ and for that reason  $\bigcap_{k=1}^m \bar{\B}(x_k, t)
\neq \emptyset \implies \{x_1, ..., x_m\} \in \mathcal{N}(\mathcal{V}^t)$.
Therefore as $t$ increases, no simplices are removed and, maybe, more can be
added. 

Remember that $\beta_0(K) = dim Z_0(K) - dim B_0(K)$. We already know that
$Z_0(K) = C_0(K)$, so its dimension do not change. Otherwise, observe that
$C_1(\mathcal{N}(\mathcal{V}^s)) \subset C_1(\mathcal{N}(\mathcal{V}^t))$, as we
proved above. In that sense, no element can be removed from
$B_0(\mathcal{N}(\mathcal{V}^s))$ and hence its dimension cannot decrease (and
it can increase if we add new elements to it, when new simplices in
$C_1(\mathcal{N}(\mathcal{V}^t))$ are not mapped in the existing images). We
conclude $t \mapsto \beta_0(\mathcal{N}(\mathcal{V}^t))$ is non-increasing,
then $t \mapsto \beta_0(X^t)$ is non-increasing. 

We can use the same argument to prove $t \mapsto \beta_0^{Rips}(t)$ is
non-increasing. When $t$ increases, we only can add edges, but not remove, as
the considered difference increases (note the condition to add edges is $x_i,
x_j$ such that $||x_i - x_j|| \le 2t$).  
    
\end{proof}

\noindent\linia

The computational exercises (44 - 46) can be found in the
Github\footnote{\url{github.com/lucasmoschen/topological-data-analysis/blob/main/tutorials/tutorial-2.ipynb}}

\begin{exercise}
    In the notebook is given a subset of $\R^4$ with 200 elements. It has been
    sampled on a famous 2-dimensional object. Compute the Betti curves of its
    Rips complex on $[0, 1]$. Can you recognize which surface it is?
\end{exercise}

\noindent\linia

\begin{exercise}
    In the notebook is given a collection of images from
    \url{https://www.cs.columbia.edu/CAVE/software/softlib/coil-20.php}. It
    consists of 20 objects, for each of which 72 pictures have been taken.
    Each image has 128×128 pixels. Embed each collection of 72 images in R
    128×128 , and compute the Betti curves of the corresponding Rips complex.
\end{exercise}

\noindent\linia

\begin{exercise}
    We are given the data of this
    paper\footnote{\url{https://pubmed.ncbi.nlm.nih.gov/26812805/}}. It
    consists in 14 correlation matrices, each matrix representing correlations
    between the components of a protein. Transform the matrices of
    correlations into matrices of distances. Then, compute the 1-Betti curves
    of the Rips complex for each of these matrices of distances. Compare the
    Betti curves of the different proteins. Do you recognize two different
    types of proteins (open and closed )?
\end{exercise}