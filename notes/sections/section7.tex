\subsection{Important definitions}

\begin{definition}
    For every $t \ge 0$, the $t$-thickening of the set $X$, denoted $X^t$, is
    the set of points of the ambient space with distance at most $t$ from $X$:
    $$X^t = \{y \in \R^n, \exists x \in X, ||x - y|| \le t\}.$$
    Equivalently, $X^t$ can be seen as a union of closed balls centered around
    every point of $X$.
\end{definition}

\begin{definition}[Hausdorff distance]
    Let $X$ be any subset of $\R^n$. The function distance to $X$ is the map 
    \begin{align*}
        \text{dist }(\cdot, X) : \R^n &\rightarrow \R \\
        y &\mapsto \text{dist }(y, X) = \inf\{||y - x||, x \in X\}
    \end{align*}
    A projection of $y \in \R^n$ on $X$ is a point $x \in X$ which attains
    this infimum. If such a point $x$ exists and is unique, we denote it
    $\text{proj }(y, X)$. 

    Then the Hausdorff distance cna be written as 
    $$
    d_H(X, Y) = \max(\sup_{y\in Y} \text{dist }(y, X), \sup_{x\in X} \text{dist }(x, Y))$$
\end{definition}

\begin{definition}
    Let $X$ be any subset of $\R^n$. The medial axis of $X$ is the subset
    $med(X) \subset \R^n$ which consists of points $y \in \R^n$ that admit at
    least two projections on $X$:
    $$
    med(X) = \{y \in \R^n, \exists x, x' \in X, x \neq x', ||y - x|| = ||y - x'|| = dist(y, X)
    .$$
\end{definition}

\begin{definition}
    Now, we define the reach of $X$ as its proximity from its medial axis:
    $$
    reach(X) = \inf\{dist(y, X), y \in med (X)\} = \inf\{||x - y||, x \in X, y \in med (X)\}
    $$
    Equivalently
    $$
    reach(X) = \sup\{t \ge 0, X^t \cap med (X) = \emptyset\}
    $$
\end{definition}

\begin{definition}
    Let $X$ be a topological space, and $U = \{U_i\}_{1\le i\le N}$ a cover of
    $X$. The nerve of $U$ is the simplicial complex with vertex set $\{1, ...,
    N\}$ and whose m-simplices are the subsets $\{i_1, ..., i_m\} \subset \{1,
    ..., N\}$ such that $\cap_{k=0}^m U_{i_k} \neq \emptyset$. It is denoted $\mathcal{N}(U)$.
\end{definition}

\begin{definition}
    Let $t \ge 0$ and consider the collection $\mathcal{V}^t = \{
    \bar{\B}(x,t), x \in X\}$. Its nerve is denoted Cech$^t(X)$ and is called the Cech complex of $X$ at time $t$.
\end{definition}

\begin{definition}
    Given a graph $G$, the corresponding clique complex is the simplicial
    complex whose vertices are the vertices of $G$, and whose simplices are
    the sets of vertices of the cliques of $G$. Some authors also call it the
    expansion of $G$
    .
\end{definition}

\begin{definition}
    The Rips complex of $X$ at time $t$ is the clique complex of the graph
    $G^t$ defined above. We denote it Rips$^t(X).$
\end{definition}

\subsection{Exercises}

\begin{exercise}
    Prove that, when $X$ is closed and bounded, a projection always exists. A set is bounded if there exists $R > 0$ such that $X \subset \B(R, 0)$.
\end{exercise}

Let $X$ a closed and bounded subset of $\R^n$. So it's compact by the
Heine-Borel theorem. The map $y \mapsto ||y - x||$ is continuos because for $y
\in \R^n$ and $\epsilon > 0$, if we take $\delta = \epsilon$, and $||y - w|| <
\delta, $
$$
\epsilon > ||y - w|| = ||(y - x) - (w - x)|| \ge |||y-x|| - ||w-x|||
$$

By the Weierstrass theorem, the function $y \mapsto ||y-x||$ has a global
minimum in $X$, that is, there exist $x^* \in X$ such that $||y - x^*|| \le
||y - x||, \forall x \in X$. So the infimum is well defined and a projection
always exist. However the uniqueness is not guaranteed. 

\noindent\linia

\begin{exercise}
    Let $||\cdot||+{\infty}$ be the sup norm of function $f : \R^n \to \R^m:
    ||f||_{\infty} = \sup_{x \in \R^n} ||f(x)||$. Prove that $d_H(X, Y) = ||\text{dist}(\cdot, X) - \text{dist}(\cdot, Y)||_{\infty}$.
\end{exercise}

We have that 
\begin{equation*}
    \begin{split}
        ||\text{dist }(\cdot, X) - \text{dist }(\cdot, Y)||_{\infty} &= \sup_{x \in \R^n} |\text{dist}(x, X) - \text{dist}(x, Y)| \\
        &\ge \sup_{x \in \R^n} (\text{dist}(x, X) - \text{dist}(x, Y)) \\
        &\ge sup_{y \in Y} dist(y,X)
    \end{split}
\end{equation*}
given that $Y \subset \R^n$ and $dist(y,Y) = 0$. Also

\begin{equation*}
    \begin{split}
        ||\text{dist }(\cdot, X) - \text{dist }(\cdot, Y)||_{\infty} &= \sup_{x \in \R^n} |\text{dist}(x, X) - \text{dist}(x, Y)| \\
        &\ge \sup_{x \in \R^n} (\text{dist}(x, Y) - \text{dist}(x, X)) \\
        &\ge sup_{x \in X} dist(x,Y)
    \end{split}
\end{equation*}
In special, $||\text{dist }(\cdot, X) - \text{dist }(\cdot, Y)||_{\infty} \ge
d_H(X,Y)$. 

The other inequality can be seen as follows. If $dist(x,Y) \le k, \forall x \in X$, then for all $w \in \R^n$ and $x \in
X$, $$dist(w,Y) \le dist(w,x) + dist(x,Y) \le dist(w,x) + k$$ and hence
$dist(w,Y) \le dist(w,X) + k$, taking the infimum. Likewise, if $dist(y,X) \le
k, \forall y \in Y$ we obtain $dist(w, X) \le dist(w,Y) + k$. Then 
$|dist(w,X) - dist(w,Y)| \le k$, for all $w \in \R^n$.
In particular $d_H(X,Y) \ge dist(x,Y), \forall x \in X$ and $d_H(X,Y) \ge
dist(y,X), \forall y \in Y$. By the fact we have proved 
$$
d_H(X,Y) \ge \sup_{w \in R^n}|dist(w,X) - dist(w,Y)| = ||dist(\cdot, X) - dist(\cdot, Y)||_{\infty}
$$
We conclude $d_H(X,Y) = ||dist(\cdot, X) - dist(\cdot, Y)||_{\infty}$. 

\noindent\linia

\begin{exercise}
    Let $X, Y$ be two closed and bounded subsets of $\R^n$. Show that for
    every $t \ge 0$, the thickenings satisfy 
    $$
    d_H(X^t, Y^t) \le d_H(X, Y).
    $$ 
    Give an example for which $d_H(X^t, Y^t) < d_H(X, Y)$.
\end{exercise}

Let's do it by steps, because dealing with infimum and supremum can be trick.
Let $t \ge 0$ and denote $dist$ by $d$. 

\begin{enumerate}
    \item Let's prove $d(w,Y^t) \le d(w,Y) - t, \forall w \in X^t/Y^t$. 
    
    Take $y \in Y$ and denote $\beta = ||w - y|| > t$. Take the point $\alpha
    w + (1 - \alpha)y$ such that, for some $\alpha \in [0,1]$, $t = ||\alpha y
    + (1-\alpha)y - y|| = \alpha\beta$, that is, $\alpha = t/\beta < 1$. So we
    will have $||w - \alpha w - (1 - \alpha)y|| = (1 - \alpha)\beta = \beta -
    t$. It implies $d(w,Y^t) \le d(w,Y) - t$. After we will se the restriction
    over the domain of $w$ is not so restrictive. 

    \item  $\sup_{w \in X^t} d(w, Y) \le \sup_{x \in X} d(x,Y) + t$ 
    
    If $w \in X^t$, there exists $x \in X$ such that $||w - x|| \le t$.
    So $d(w, Y) \le d(x,Y) + t \le \sup_{x \in X} d(x,Y) + t$. It values for
    all $w$, then $\sup_{w \in X^t} d(w, Y) \le \sup_{x \in X} d(x,Y) + t$.

    \item $\forall w \in X^t, d(w, Y^t) \le \sup_{w \in X^t} d(w, Y) - t$
    
    For every $w \in X^t/Y^t, d(w, Y^t) \le \sup_{w \in X^t} d(w, Y) -
    t$. If $w \in Y^t \cap X^t, d(w,Y^t) = 0$, so this inequality hols for
    every $w \in X^t$, since $\sup_{w \in X^t} d(w, Y) \ge t$.
    
    \item Based on the last two points, $\sup_{w \in X^t} d(w, Y^t) \le \sup_{w \in
    X^t} d(w, Y) - t \le \sup_{x \in X} d(x, Y)$. 

    \item We can show similarly that $\sup_{z \in Y^t} d(z, X^t) \le \sup_{y
    \in Y} d(y, X)$. 

    \item Therefore, the last two points imply $d_H(X^t, Y^t) \le d_H(X,Y)$. 

\end{enumerate}

\textbf{Example}

Consider the following construction. Let $C_0$ be the origen and $C_i$ be the
circumference of center 0 and radius $\sum_{j=0}^{i-1} 2^{-j} = 2 - 2^{1-i}$.
Each time the adicional value to the radius of the next circle is decreasing
by half. All the circles are contained in $\B(0,2)$. We will define $$X =
\bigcup_{i=0}^{\infty} C_{2i} \text{ and } Y =
\bigcup_{i=0}^{\infty} C_{2i + 1}$$

\begin{figure}[H]
    \centering
    \includegraphics[width = 0.3\textwidth]{../images/circle-inside-circle.png}
\end{figure}

I claim that $d_H(X,Y) = 1$. For every $x \in C_i$, we have that $d(x,Y) =
2^{-i}$, because the closest point in $Y$ is in $C^{i+1}$. We can argue the
same for $d(y,X)$. Therefore $d_H(X,Y) = \sup_i 2^{-i} = 1$. Consider now
$X^t$ and $Y^t$ for some $t > 0$. We are creating rings with thickness $2t$.
Some rings of different sets are being joined. Let's calculate $d_H(X^t,
Y^t)$. In general, $d(x,Y^t)$ will decreased, be unaltered (when we pick some
point in a border of $C_i^t$ that maps to $C_{i+1}^t$ or go to $0$, when two
rings join. But it shall not increase. However if we take the origin, $d(0,Y)
= 1 - t$, but no other can have a greater distance. Thus, $d_H(X^t, Y^t) = 1-t
< 1 = d_H(X, Y)$. 

\noindent\linia

\begin{exercise}
    Show that the Hausdorff distance is equal to
    $$inf\{t \ge 0, X \subset Y^t \text{ and } Y \subset X^t\}$$
\end{exercise}

Let $t$ such that $X \subset Y^t$ and $Y \subset X^t$. Take $x \in X$ and
consider $d(x,Y)$. By the choice of $t, x \in Y^t$, what implies $d(x,Y) \le
t$, because we can find $y \in Y$ such that $||x - y|| \le t$. As $x$ is
arbitrary, $\sup_{x \in X} d(x,Y) \le t$. The same can be told about $\sup_{y
\in Y} d(y, X)$. That implies $d_H(X,Y) \le t$. As we took arbitrary $t$,
applying the infimum we obtain 
$$d_H(X,Y) \le \inf\{t \ge 0, X \subset Y^t, Y \subset X^t\}.$$
Suppose the above inequality is restricted ($<$). Then there is $\epsilon > 0$
such that $d_H(X, Y) + \epsilon$ is still less. Define $s = d_H(X,Y) + \epsilon$. Then $s > d(x,Y), \forall x \in X$ and $s > d(y,
X), \forall y \in Y$. As $s > d(x,Y)$, there is $y \in Y$ such that $||x - y||
< s \implies x \in \bar{\B}(x,y) \implies x \in Y^s$. This holds for every $x
\in X$, so $X \subset Y^s$. Likewise we prove $Y \subset X^s$. That is a
contradiction, since $s < \inf\{t \ge 0, X \subset Y^t, Y \subset X^t\}$. We
conclude that $d_H(X, Y) = \inf\{t \ge 0, X \subset Y^t, Y \subset X^t\}.$  

\noindent\linia 

\begin{exercise}
    Compute the reach of the following subsets of $R^2$:
    \begin{enumerate}
        \item the set $\{(0, n), n \in \Z\},$
        \item the segment $\{(t, 0), t \in [0, 1]\},$
        \item the unit circle with origin $\sphere_1 \cup \{(0, 0)\}$,
        \item the square $\{(x, y) \in R^2, \max\{|x|, |y|\}  
        = 1\},$
        \item the ellipse $\{(x_1, x_2) \in \R^2, (\frac{x_1}{a})^2 +
        (\frac{x_2}{b})^2 = 1 \}$
    \end{enumerate}
\end{exercise}

\begin{enumerate}
    \item Define this set as $Z$. So $reach(Z) = \inf\{d(y,Z), y \in
    med(Z)\}$. We have that $med(Z) = \{(x,n+1/2), x \in \R, n \in \Z\}$, the
    bisector os two sequential points. For each point $y$ in one of these
    lines, $dist(y,Z) = ||(x,n+1/2) - (0,n)|| = ||(x,1/2)|| = \sqrt{x^2 +
    1/4}$. The infimum over $y$ occurs when $x = 0$ and we conclude $reach(Z)
    = 1/2$. 

    \item Denote this set as $I$. Take $y = (y_1, y_2) \in \R^2$. We want
    the values on $I$ which distance to $y$ is $d(y,I) = \inf\{\sqrt{(y_1 -
    t)^2 + y_2^2}, t \in [0,1] \} = \min\{\sqrt{(y_1 -
    t)^2 + y_2^2}, t \in [0,1] \}$, since $I$ is closed and bounded, there exists a projection. We must
    minimize $(y_1 - t)^2$ in the interval $[0,1]$, but this clearly have only
    one solution. We conclude $med(I) = \emptyset$ and $reach(I) = \infty$. 
    
    \item We have that $med(\sphere_1 \cap 0) = \partial\B(0,1/2)$. For each
    $y \in \partial\B(0,1/2), d(y,Z) = 1/2$, the distance to the origin.
    Therefore $reach(\sphere \cap 0) = 1/2$. 

    \item The medial axis is similar to the circle, a unique point in the
    origin. Then $reach(square) = d(0,square) = 1$. 

    \item We already know the open interval is the median axis of the ellipse. 

\end{enumerate}

\noindent\linia

\begin{exercise}
    Compute the reaches of the subsets of Exercise 39.
\end{exercise}

\noindent\linia

\begin{exercise}
    Verify that the clique complex of a graph is a simplicial complex. If the
    graph contains $n$ vertices, give an upper bound on the number of
    simplices of the clique complex.
\end{exercise}

\noindent\linia

\begin{exercise}
    Improve the previous proposition as follows: Cech$^t(X) \subset $
    Rips$^t(X) \subset $ Cech$^{ct}(X)$, where $c = \sqrt{\frac{2n}{n+1}}$.
\end{exercise}

\newpage